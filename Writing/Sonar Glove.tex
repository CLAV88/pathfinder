% replace article with report for a separate title page and abstract page
\documentclass{article}

% amsmath package, useful for mathematical formulas
\usepackage{amsmath}

% amssymb package, useful for mathematical symbols
\usepackage{amssymb}

% graphicx package, useful for including eps and pdf graphics
% include graphics with the command \includegraphics
\usepackage{graphicx}

% cite package, to clean up citations in the main text. Do not remove.
\usepackage{cite}

\usepackage{float}
\usepackage{color} 

% Use doublespacing - comment out for single spacing
\usepackage{setspace} 
\doublespacing

% Indents the first paragraph of each new section (not indented by default)
\usepackage{indent first}

% Text layout
\topmargin 0.0cm
\oddsidemargin 0.5cm
\evensidemargin 0.5cm
\textwidth 16cm 
\textheight 21cm

% Bold the 'Figure #' in the caption and separate it with a period
% Captions will be left justified
\usepackage[labelfont=bf,labelsep=period,justification=raggedright]{caption}

% Uncomment 3 lines below to remove brackets from numbering in References
%\makeatletter
%\renewcommand{\@biblabel}[1]{\quad#1.}
%\makeatother

\pagestyle{plain}

%% INCLUDE ALL MACROS BELOW

%% END MACROS SECTION

\title{Project Pathfinder:
\\
A Sonar Glove Assisting in Navigation for the Visually Impaired
}
\author{Neil Movva \\ Grade 12, The Harker School}
% comment out the line below to show the current date
% \date{}


\begin{document}

\maketitle


\begin{abstract}

There is a dearth of technological solutions tailored to blind users. The author has attempted to address this need by designing, characterizing, and refining an assistive navigational device for the blind. Fundamentally, the device is a piece of wearable technology that incorporates an ultrasonic transceiver, a lightweight microcontroller, and a haptic interface to convey environmental information to the user. Additional sensors, such as an accelerometer, provide more contextual data that helps the device behave more naturally. This paper will discuss in detail the concepts underlying the device, as well as the considerations that go along with it, such as low-power design, optimal haptic feedback patterns, small footprint layout and manufacturing, etc.

\end{abstract}


\end{document}